%% start of file `template.tex'.
%% Copyright 2006-2013 Xavier Danaux (xdanaux@gmail.com).
%
% This work may be distributed and/or modified under the
% conditions of the LaTeX Project Public License version 1.3c,
% available at http://www.latex-project.org/lppl/.


\documentclass[11pt,letterpaper, sans]{moderncv}        % possible options include font size ('10pt', '11pt' and '12pt'), paper size ('a4paper', 'letterpaper', 'a5paper', 'legalpaper', 'executivepaper' and 'landscape') and font family ('sans' and 'roman')

\moderncvstyle{oldstyle}

\moderncvcolor{grey}

% character encoding
\usepackage[utf8]{inputenc}

% adjust the page margins
\usepackage[scale=0.75]{geometry}
%\setlength{\hintscolumnwidth}{3cm}                % if you want to change the width of the column with the dates
%\setlength{\makecvtitlenamewidth}{10cm}           % for the 'classic' style, if you want to force the width allocated to your name and avoid line breaks. be careful though, the length is normally calculated to avoid any overlap with your personal info; use this at your own typographical risks...

% personal data
\name{Juan}{Barbosa}
\address{Calle 126 \# 52 a 92}{Bogot\'a}{Colombia}% optional, remove / comment the line if not wanted; the "postcode city" and and "country" arguments can be omitted or provided empty
\phone[mobile]{+57~(316)~794~6510}                  % optional, remove / comment the line if not wanted
\email{js.barbosa10@uniandes.edu.co}                               % optional, remove / comment the line if not wanted
\homepage{www.github.com/jsbarbosa}                         % optional, remove / comment the line if not wanted
%\extrainfo{additional information}                 % optional, remove / comment the line if not wanted

%----------------------------------------------------------------------------------
%            content
%----------------------------------------------------------------------------------
\begin{document}
%-----       letter       ---------------------------------------------------------
% recipient data
\recipient{James Purvis}{CERN\\CH-1211 Geneva 23\\
	Switzerland}
\date{October 25, 2016}
\opening{Dear Mr. James,}
\closing{Thanking you Sir,}
%\enclosure[Attached]{curriculum vit\ae{}}          % use an optional argument to use a string other than "Enclosure", or redefine \enclname
\makelettertitle

I just heard about the short-term internship at The European Organization for Nuclear Research, and I find it a very nice opportunity to get in touch with the very edge of science. I am a student major in physics and chemistry at \textit{Universidad de los Andes}. As an undergraduate student of these natural sciences, I am very interested in understanding the atomic and subatomic world. I constantly wonder who could be better teacher than CERN, not only because of your latests discoveries such as the Higgs boson, but all the break throws made in history. 

Ever since I was little, I had a very special interest in programming and physics. The whole idea of understanding everything around me as a finite set of principles sounded great, and I think that is where both of my interests came together. After some time, lectures and a lot of experiments, I got a descent knowledge in the language of nature and Python, both of which you are very interested in. Because of my concern in making knowledge accessible to everyone -as you do- I began the GitHub project, an effort to make public all my work. So far I have made two major softwares, the first one was intended to solve the problem of finding specific molecules in GC-MS data by making comparison with mass spectrums in public databases. The second one was made to simplify the data acquisition in the modern physics laboratory at my \textit{Alma mater}. CERN is one of top organizations to produce experimental data, which means there is a lot of analysis to do, and a bunch of space to improve. Data analysis is my passion, solving real-world problems such as the one you have is what I love. 


I am looking forward to discuss any further details, feel free to contact me. You can find me on Skype, GitHub and email. I repeat, CERN a great place to learn.

\makeletterclosing

\end{document}
